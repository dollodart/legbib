\documentclass[english]{article}
\usepackage{legbib}
\
\addbibresource{test.bib}

\begin{document}

At first there was \cite{memo1}. Then there was \cite{memo2} criticizing
\cite{memo1}. Then there was \cite{memo3} criciticsing the criticisms in
\cite{memo2} of \cite{memo1}. The Judge must now rule on \cite{memo1},
\cite{memo2}, and \cite{memo3}.

Let's check how well it's formatted:

\begin{itemize}
\item \cite{memo1}.
\item Mot.\ Dism..
\item \cite{memo2}.
\item Resp.\ Opp.\ Mot.\ Dism..
\item \cite{memo3}.
\item Rep.\ Supp.\ Mot.\ Dism..
\end{itemize}

Biblatex by default absorbs following periods as part of its punctuation buffer. While some stylists agree that an abbreviation dot does ``double duty'' as a sentence period, this isn't logical punctuation. So that may be turned off with \texttt{\string\bibsentence} in the citation driver.

\printbibliography

\end{document}
